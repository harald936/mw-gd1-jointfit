
% Auto-generated summary for mw-gd1-jointfit
\section*{Milky Way RC baseline (axisymmetric; running-median fit)}

\textbf{Best-fit (MLE) parameters} (this work, RC-only):
\begin{align*}
v_h &= 202.6~\mathrm{km\,s^{-1}}, &
r_h &= 10.00~\mathrm{kpc}, &
\gamma &= 1.60,\\
q_z &\approx 0.90~\text{(held for RC-only)},\quad
s_d &= 1.30,\quad
a &= 5.30~\mathrm{kpc},\quad
s_b &= 1.30.
\end{align*}

\textbf{Interpretation}:
\begin{itemize}
\item $v_h$ sets the halo contribution to $V_c$.
\item $r_h$ is the halo scale radius (where the RC transitions).
\item $\gamma$ is the inner gNFW slope (higher $\gamma$ = cuspier). Here we softly anchor $\gamma$ to the NumPyro RC-only posterior.
\item $q_z$ is the halo flattening (vertical shape). RC alone is weakly sensitive; GD-1 will sharpen this.
\item $s_d$, $a$ adjust disk mass and radial scale; $s_b$ scales the bulge (helps with baryon–halo degeneracy).
\end{itemize}

\textbf{Method in brief}: We fit the running-median RC over $6{-}18~\mathrm{kpc}$
with a disk+bulge+gNFW halo using global+local optimization, a Solar-circle anchor
$V_c(8.2~\mathrm{kpc})\approx230~\mathrm{km\,s^{-1}}$, and soft priors on baryons and halo.
This captures the broad trend relevant for halo constraints while avoiding overfitting non-axisymmetric wiggles.

\textbf{Next step (joint fit)}: Include GD-1 (Gaia DR3) to constrain $q_z$.
The stream’s 3D track and proper motions are sensitive to the vertical force, so combining RC (in-plane)
with GD-1 (vertical sensitivity) breaks degeneracies and tightens the halo shape.
