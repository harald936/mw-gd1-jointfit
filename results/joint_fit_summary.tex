\paragraph{Joint rotation-curve + GD-1 fit.}
We fit an axisymmetric Milky Way model with a Plummer bulge, a Miyamoto–Nagai disk,
and a generalized NFW halo (flattened). Using Gaia DR3 rotation-curve points (5–25 kpc)
together with the GD-1 stream selection from Gaia DR3, we obtain a circular-velocity
profile consistent with \(V_c(R_\odot)\approx 230~\mathrm{km\,s^{-1}}\).
Maximum-likelihood halo parameters are
\(v_h=\underline{XXX}~\mathrm{km\,s^{-1}},\ r_h=\underline{XX.X}~\mathrm{kpc},\ \gamma=\underline{X.XX},\ q_z=\underline{0.8X}\),
with credible intervals reported separately. Including GD-1 significantly sharpens the
halo flattening constraint (favoring an oblate halo, \(q_z<1\)) and reduces the
\(v_h\)–\(r_h\) degeneracy relative to a rotation-curve–only fit.
